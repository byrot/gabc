%%%Psaume préparé pour les tons : 7a,
\verset{Dómini est terra, et pleni\acd{tú}do \acc{e}ius : * orbis terrárum, et univérsi qui hábi\acb{\prab{tant}} \praa{in} \aca{e}o.}
\verset{Quia ipse super mária fun\acd{dá}vit \acc{e}um : * et super flúmina præpa\acb{\prab{rá}}\praa{vit} \aca{e}um.}
\verset{Quis ascéndet in \acd{mon}tem \acc{Dó}mini ? * aut quis stabit in loco \acb{\prab{san}}\praa{cto} \aca{e}ius ?}
\verset{Innocens mánibus et mundo corde, \+\ qui non accépit in vano \acd{á}nimam \acc{su}am, * nec iurávit in dolo \acb{pró}\prab{xi}\praa{mo} \aca{su}o.}
\verset{Hic accípiet benedicti\acd{ó}nem a \acc{Dó}mino : * et misericórdiam a Deo, salu\acb{\prab{tá}}\praa{ri} \aca{su}o.}
\verset{Hæc est generátio quæ\acd{rén}tium \acc{e}um, * quæréntium fáciem \acb{\prab{De}}\praa{i} \aca{Ia}cob.}
\verset{Attóllite portas, príncipes, vestras, \+\ et elevámini, portæ \acd{æ}ter\acc{ná}les : * et intro\acb{í}\prab{bit} \praa{Rex} \aca{gló}riæ.}
\verset{Quis est iste Rex glóriæ ? \+\ Dóminus \acd{for}tis et \acc{pot}ens : * Dóminus \acb{pot}\prab{ens} \praa{in} \aca{pr\'\ae}lio.}
\verset{Attóllite portas, príncipes, vestras, \+\ et elevámini, portæ \acd{æ}ter\acc{ná}les : * et intro\acb{í}\prab{bit} \praa{Rex} \aca{gló}riæ.}
\verset{Quis est \acd{i}ste Rex \acc{gló}riæ ? * Dóminus virtútum ipse \acb{\prab{est}} \praa{Rex} \aca{gló}riæ.}
